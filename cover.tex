% -*-latex-*-
% 
% For questions, comments, concerns or complaints:
% thesis@mit.edu
% 
%
% $Log: cover.tex,v $
% Revision 1.8  2008/05/13 15:02:15  jdreed
% Degree month is June, not May.  Added note about prevdegrees.
% Arthur Smith's title updated
%
% Revision 1.7  2001/02/08 18:53:16  boojum
% changed some \newpages to \cleardoublepages
%
% Revision 1.6  1999/10/21 14:49:31  boojum
% changed comment referring to documentstyle
%
% Revision 1.5  1999/10/21 14:39:04  boojum
% *** empty log message ***
%
% Revision 1.4  1997/04/18  17:54:10  othomas
% added page numbers on abstract and cover, and made 1 abstract
% page the default rather than 2.  (anne hunter tells me this
% is the new institute standard.)
%
% Revision 1.4  1997/04/18  17:54:10  othomas
% added page numbers on abstract and cover, and made 1 abstract
% page the default rather than 2.  (anne hunter tells me this
% is the new institute standard.)
%
% Revision 1.3  93/05/17  17:06:29  starflt
% Added acknowledgements section (suggested by tompalka)
% 
% Revision 1.2  92/04/22  13:13:13  epeisach
% Fixes for 1991 course 6 requirements
% Phrase "and to grant others the right to do so" has been added to 
% permission clause
% Second copy of abstract is not counted as separate pages so numbering works
% out
% 
% Revision 1.1  92/04/22  13:08:20  epeisach

% NOTE:
% These templates make an effort to conform to the MIT Thesis specifications,
% however the specifications can change.  We recommend that you verify the
% layout of your title page with your thesis advisor and/or the MIT 
% Libraries before printing your final copy.
\title{Calibration and Fusion of Stereoscopic Cameras and Optical Range Finder Sensors for Zero Gravity Targets Inspection}

\author{Gabriel P. Urbain}
% If you wish to list your previous degrees on the cover page, use the 
% previous degrees command:
%       \prevdegrees{A.A., Harvard University (1985)}
% You can use the \\ command to list multiple previous degrees
%       \prevdegrees{B.S., University of California (1978) \\
%                    S.M., Massachusetts Institute of Technology (1981)}
\department{Department of Electronics, Optronics and Signal Processing (DEOS)}

% If the thesis is for two degrees simultaneously, list them both
% separated by \and like this:
% \degree{Doctor of Philosophy \and Master of Science}
\degree{Master of Science in Aerospace Engineering}

% As of the 2007-08 academic year, valid degree months are September, 
% February, or June.  The default is June.
\degreemonth{October}
\degreeyear{2014}
\thesisdate{October 23, 2014}

%% By default, the thesis will be copyrighted to MIT.  If you need to copyright
%% the thesis to yourself, just specify the `vi' documentclass option.  If for
%% some reason you want to exactly specify the copyright notice text, you can
%% use the \copyrightnoticetext command.  
%\copyrightnoticetext{\copyright IBM, 1990.  Do not open till Xmas.}

% If there is more than one supervisor, use the \supervisor command
% once for each.
\supervisor{Alvar Saenz-Otero}{Principal Research Scientist, Space Systems Laboratory, MIT}
\supervisor{Daniel Alazard}{Professor, Department of Mathematics, Computer Science and Control (DMIA), ISAE Supaero}

% This is the department committee chairman, not the thesis committee
% chairman.  You should replace this with your Department's Committee
% Chairman.
\chairman{moi}{moi}
% Make the titlepage based on the above information.  If you need
% something special and can't use the standard form, you can specify
% the exact text of the titlepage yourself.  Put it in a titlepage
% environment and leave blank lines where you want vertical space.
% The spaces will be adjusted to fill the entire page.  The dotted
% lines for the signatures are made with the \signature command.
\maketitle

% The abstractpage environment sets up everything on the page except
% the text itself.  The title and other header material are put at the
% top of the page, and the supervisors are listed at the bottom.  A
% new page is begun both before and after.  Of course, an abstract may
% be more than one page itself.  If you need more control over the
% format of the page, you can use the abstract environment, which puts
% the word "Abstract" at the beginning and single spaces its text.

%% You can either \input (*not* \include) your abstract file, or you can put
%% the text of the abstract directly between the \begin{abstractpage} and
%% \end{abstractpage} commands.

% First copy: start a new page, and save the page number.
\cleardoublepage
% Uncomment the next line if you do NOT want a page number on your
% abstract and acknowledgments pages.
% \pagestyle{empty}
\setcounter{savepage}{\thepage}
\begin{abstractpage}
In many areas of robotics, vision is becoming more and more common in applications such as localization, automatic map construction, autonomous navigation, path following, inspection, monitoring or risky situation detection. With the increasing performances of embedded computers and the development of faster algorithms in the last few years, multi-sensors data fusion is considered as an opportunity to take better advantage of different sensors features to stretch the limits. This project aims at implementing a multi-sensor data fusion algorithm involving two stereoscopic cameras and a Time-of-Flight camera (\gls{ToF}) on in-space nano-satellites called SPHERES.\\
This document is the result of five-month internship at the MIT SSL, USA as part of the final project of a double Master degree in Aerospace Engineering at ISAE, France, and in Electrical Engineering at UMONS, Belgium. The first chapter introduces the goal and the context of the project. The second chapter is dedicated to the theoretical aspect and aims at summarizing the required mathematical background and development. A third chapter analyzes concretely the implementation and finally, the results of two different experiment sets will be detailed in the fourth chapter before concluding.
\end{abstractpage}

% Additional copy: start a new page, and reset the page number.  This way,
% the second copy of the abstract is not counted as separate pages.
% Uncomment the next 6 lines if you need two copies of the abstract
% page.
% \setcounter{page}{\thesavepage}
% \begin{abstractpage}
% % $Log: abstract.tex,v $
% Revision 1.1  93/05/14  14:56:25  starflt
% Initial revision
% 
% Revision 1.1  90/05/04  10:41:01  lwvanels
% Initial revision
% 
%
%% The text of your abstract and nothing else (other than comments) goes here.
%% It will be single-spaced and the rest of the text that is supposed to go on
%% the abstract page will be generated by the abstractpage environment.  This
%% file should be \input (not \include 'd) from cover.tex.
In this thesis, I designed and implemented a compiler which performs
optimizations that reduce the number of low-level floating point operations
necessary for a specific task; this involves the optimization of chains of
floating point operations as well as the implementation of a ``fixed'' point
data type that allows some floating point operations to simulated with integer
arithmetic.  The source language of the compiler is a subset of C, and the
destination language is assembly language for a micro-floating point CPU.  An
instruction-level simulator of the CPU was written to allow testing of the
code.  A series of test pieces of codes was compiled, both with and without
optimization, to determine how effective these optimizations were.

% \end{abstractpage}

\cleardoublepage

\section*{Acknowledgments}
Je remercie toutes les personnes remarquables qui m'ont aid\'{e} tout au long de ce stage, \`{a} commencer par Dr. Alvar Saenz-Otero, qui m'a accueilli dans son service et a toujours sacrifi\'{e} un temps pr\'{e}cieux afin de m'orienter, tous les membres du SSL pour leur aide et leurs conseils, le professeur Daniel Alazard pour sa participation dans le d\'{e}marrage du projet, Marina G. March qui a entrepris cette exp\'{e}rience \`{a} mes c\^{o}t\'{e}s ainsi que tous mes professeurs \`{a} Toulouse et \`{a} Mons, et plus particuli\`{e}rement St\'{e}phanie Lizy-Destrez et Thierry Dutoit qui m'ont aid\'{e} personnellement dans tous mes projets.\\\\
J'aimerais aussi exprimer toute ma gratitude envers David Steinberg, Alice Malter, Jean-Fran\c{c}ois Toubeau et mon p\`{e}re pour la relecture de ce document mais aussi GDF Suez, l'Universit\'{e} de Mons et la Fondation Fernand Lazard pour le soutien financier qui m'a permis de mener \`{a} bien ce stage aux Etats-Unis.\\\\
Enfin, un merci tout particulier va \`{a} mes parents, ma soeur et mes amis \`{a} Mons et Toulouse qui, \`{a} travers leur soutien, leur \'{e}nergie et leur sourire, ont transform\'{e} mes six ann\'{e}es universitaires en une exp\'{e}rience inoubliable.

%%%%%%%%%%%%%%%%%%%%%%%%%%%%%%%%%%%%%%%%%%%%%%%%%%%%%%%%%%%%%%%%%%%%%%
% -*-latex-*-
